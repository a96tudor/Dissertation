\chapter{Feature vector structure} \label{Appendix: Feature-vector}
	
	There are 13 features in each feature vector used in this implementation. Table \ref{Table: prep/features} shows a breakdown of the these features and there meaning.
	\begin{longtable}{| p{.12\textwidth} || p{.22\textwidth} | p{.22\textwidth} | p{.22\textwidth} | p{.15\textwidth} |}
	\hline
	\multirow{2}{3cm}{\textbf{Feature}} & \multicolumn{3}{c|}{\textbf{Explanation}} & \multirow{2}{2cm}{\textbf{Type}}\\
	& Process & Socket & File &\\
	\hline
	\textit{Node type} & \multicolumn{3}{c|}{The type of node used in this case (i.e. Process, Socket or File)} & \textbf{Categorical} \\
	\hline
	\textit{Neighbour type} & The closest node connected to it. Has to be one of File and Process & 
	\multicolumn{2}{c|}{Will always be Process} & \textbf{Categorical}\\
	\hline
	\textit{Edge type} & \multicolumn{3}{p{.75\textwidth}|}{The type of edge connecting the node to the neighbour. It will always be a $PROC\_OBJ$ edge, but the \textit{state} field can take multiple values, such as:  \textit{READ, WRITE, RaW, BIN, SERVER, CLIENT, NONE}.} & \textbf{Categorical}\\
	\hline
	\textit{Connected} & Whether the Process in question connects to a different machine via a Socket. & Whether the Sockets connects to an IP address other than 127.0.0.1. & Whether the File was downloaded from the web or not. & \textbf{Binary} \\
	\hline
	\textit{Neighbour connected} & \multicolumn{3}{c|}{Whether the neighbour is Connected or not.} & \textbf{Binary} \\ 
	\hline
	\textit{UID status} & Whether the uid is equal to the euid for the Process in question or not. & \multicolumn{2}{p{.50\textwidth}|}{Whether the uid is equal to the euid for the corresponding Process neighbour or not.} & \textbf{Binary} \\
	\hline
	\textit{GID status } & Whether the gid is equal to the egid for the Process in question or not. & \multicolumn{2}{p{.50\textwidth}|}{Whether the gid is equal to the gid for the corresponding Process neighbour or not.} & \textbf{Binary} \\
	\hline
	\textit{Version} & \multicolumn{3}{p{.75\textwidth}|}{The version number of the current node (i.e. how many nodes with the same ID and smaller timestamp are present in the graph.)} & \textbf{Integer} \\
	\hline
	\textit{Suspicious} & Whether the process accesses files in a restricted directory (eg: \textit{/etc/pwd, /usr/bin, /usr/lib, /bin, /lib}), whether the command starting the process contains keywords such as: \textit{sudo, chmod, usermod, groupmod, rm -rf}. & Whether the Process that initiated the Socket has the \textit{Suspicious} flag or not &Whether the file in question has a suspicious name, containing keywords such as: \textit{attack, worm, trojan, virus, etc.} & \textbf{Binary} \\
	\hline
	\textit{External} & Whether it connects to a different machine via a Socket. & Whether the Socket is to an IP address other than 127.0.0.1. & Whether the File contents are transmitted via a Socket to a different machine. & \textbf{Binary} \\
	\hline
	\textit{Degree} & \multicolumn{3}{c|}{The degree of the node in question.} & \textbf{Integer} \\
	\hline
	\textit{Neighbour degree} & \multicolumn{3}{c|}{The degree of the Neighbour} & \textbf{Integer} \\
	\hline
	\textit{Neighbour distance} & \multicolumn{3}{p{.75\textwidth}|}{
		\makecell{
			\begin{math}
			=
			\begin{cases} 
			0.0 & \text{if } \Delta t = 0\\
			\log \Delta t & \text{otherwise} 
			\end{cases} 
			\end{math}
			\\
			where $\Delta t = abs(node.timestamp - neighbour.timestamp)$ 
	} } & \textbf{Real} \\
	\hline
	\caption[Features table]{\centering Table listing and explaining the meaning of the features that are extracted for every node.}
	\label{Table: prep/features}
\end{longtable}