\documentclass[border=0.125cm]{standalone}
\usepackage{tikz}
\usetikzlibrary{positioning}
\usepackage{ifthen}
\usetikzlibrary{matrix,arrows.meta,quotes,shadows,decorations.pathreplacing,positioning,fadings}
\usepackage{cfr-lm}
\usetikzlibrary{shapes.misc}

\begin{document}
	\tikzset{
		module/.style = {
			draw,
			rounded rectangle,
			fill=white
		},
		reg module/.style = {
			draw,
			rounded rectangle,
			fill=blue!20
		},
		param/.style = {
			draw,
			circle,
			minimum size = .25cm,
			fill=white
		},
		reg param/.style = {
			draw,
			circle, 
			minimum size = .25cm,
			fill=blue!20
		},
		edge/.style = {
			->
		}
	}
	\begin{tikzpicture}
		\node[param] (input) at (0,0){\LARGE $\mathbf{X}$};
		\node[param] (kernel) at (-2, 0.8) {kernel};
		\node[reg module] (MatMul) at (-1, 2) {MatMul};
		\node[reg module] (relu) at (-1, 4) {ReLU};
		\node[reg module] (dropout) at (-2, 6) {Dropout};
		\node[param] (keep-prob) at (-3, 5) {$k_p$};
		
		\draw[edge] (input) -- (MatMul);
		\draw[edge] (kernel) -- (MatMul);
		\draw[edge] (MatMul) -- (relu);
		\draw[edge] (keep-prob) -- (dropout);
		\draw[edge] (relu) -- (dropout);
		
		\node (in) at (0, -2) {};
		\node (out) at (-2, 8) {};
		
		\draw[edge] (in) -- (input);
		\draw[edge] (dropout) -- (out); 
		
		\node () at (0,-2.1) {Ouputs of};
		\node () at (0, -2.5) {previous layer};
		
		\node () at (-2, 8.5) {Output of};
		\node () at (-2, 8.1) {convolutional layer};
 	\end{tikzpicture}
\end{document}