\documentclass[border=0.125cm]{standalone}
\usepackage{tikz}
\usetikzlibrary{positioning}
\usepackage{ifthen}
\usetikzlibrary{matrix,arrows.meta,quotes,shadows,decorations.pathreplacing,positioning,fadings}
\usepackage{cfr-lm}
\usepackage{graphicx}
\usetikzlibrary{shapes,shadows,arrows,positioning,graphs}


\colorlet{mewnol}{blue!75!cyan}%
\colorlet{allanol}{blue!50!black}%
\usepackage{verbatim}


\begin{document}

\tikzset{%
  every neuron/.style={
    circle,
    draw,
    minimum size=1.5cm
  },
  neuron missing/.style={
    draw=none, 
    scale=4,
    text height=0.333cm,
    execute at begin node=\color{black}$\vdots$
  },
}

\begin{tikzpicture}[x=1.5cm, y=1.5cm, >=stealth]

\foreach \m/\l [count=\y] in {1,2,3,4,5,6,missing,7}
  \node [every neuron/.try, neuron \m/.try] (input-\m) at (-1,2.5-\y*1.3) {};

\foreach \n/\l [count=\x] in {A,B,missing,C} {
	\ifthenelse{\equal{\n}{missing}}
		{
			\node[neuron missing] (hidden-missing) at (3, 7.0-\x*4.3) {};
		}
		{
			\node (text-\n) at (3.7, 7.0-\x*3.8) {\textbf{\small Class \n\space patterns}};
			\foreach \m/\r [count=\y] in {1,missing,2} {
				\ifthenelse{\equal{\m}{missing}}
					{
						\node[neuron missing, scale=.3] (missing-\n) at (3, 7.0-\x*3.7-\y*1.0) {};
					}
				{
					\node[every neuron] (hidden-\n-\m) at (3, 7.0-\x*3.7-\y*1.0) {
						\ifthenelse{\equal{\m}{2}}
							{$\mathbf{X}_{\n, N_{\n}}$}
							{$\mathbf{X}_{\n, \y}$}
						
					};
				}
			}
		}
}

\foreach \n/\a [count=\x] in {A,B,missing,C} {
	\ifthenelse{\equal{\n}{missing}}
		{
			\node[neuron missing](conds-missing) at (7, 3.5-\x*3.0) {};
			\node[neuron missing](out-missing) at (14, 3.5-\x*3.0) {};
		}
		{
			\node[every neuron] (conds-\n) at (7, 3.5-\x*3.0) {$G_{\n}$};
			\node[above=.5cm of conds-\n] (text-conds-\n) {$P(\mathbf{X}\mid \n)$};
			\node[every neuron] (output-\n) at (14, 3.5-\x*3.0) {};
			\node (out-arrow-\n) at (17, 3.5-\x*3.0) {};
			\node[scale=0.1] (priors-\n) at (10.5, 3.5-\x*3.0) {};
			\node[above left=.1cm of priors-\n] (priors-text-\n)  {$P(\n)$};
		}
}

\foreach \m in {1,...,7}
	\foreach \i/\a in {A,B,C}
		\foreach \j/\b in {1,2}
		{
			\draw[->] (input-\m) -- (hidden-\i-\j) ; 
			\draw[->] (hidden-\i-\j) -- (conds-\i);
			\draw (conds-\i) -- (priors-\i);
			\draw[->] (output-\i) -- (out-arrow-\i) node[above, midway] {Class \i};
			\foreach \k/\c in {A,B,C}
				\draw[->] (priors-\i) -- (output-\k);
		}
	
\draw [decorate,decoration={brace,amplitude=10pt},xshift=-4pt,yshift=0pt]
				(-2,-9.0) -- (-2,2.0) node [black,midway,xshift=-0.6cm] 
		{};
\node (text-input-1) at (-2.8, -3.3) {Input};
\node (text-input-2) at (-2.8, -3.7) {patterns};

\node (title-conds) at (7, 3.5) {\textbf{Class conditional densities}};
\node (title-conds) at (10.5, 3.5) {\textbf{Priors}};

\node (input-layer) at (-1, -12) {\textbf{Input Layer}};
\node (hidden-layer) at (3, -12) {\textbf{Hidden Pattern Layer}};
\node (summ-layer) at (8.75, -12) {\textbf{Summation Layer}};
\node (out-layer) at (14, -12) {\textbf{Output Layer}};
 \end{tikzpicture}

\end{document}